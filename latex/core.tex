%%%%%%%%%%%%%%%%%%%%%%%%%%%%%%%%%%%%%%%%%%%%%%%%%%%%%%%%%%%%%%%%%%%%%%%%%%%%%%
% `preamble.tex` (core.tex) is a file commonly used in LaTeX projects to define 
% custom configurations and packages for the document. It typically includes 
% settings that control the overall appearance and behavior of the document, 
% such as font size, margins, and the inclusion of external packages for 
% enhanced functionality.
%%%%%%%%%%%%%%%%%%%%%%%%%%%%%%%%%%%%%%%%%%%%%%%%%%%%%%%%%%%%%%%%%%%%%%%%%%%%%%
% Suppress overfull and underfull box warnings
\hbadness=10000
\vbadness=10000
%%%%%%%%%%%%%%%%%%%%%%%%%%%%%%%%%%%%%%%%%%%%%%%%%%%%%%%%%%%%%%%%%%%%%%%%%%%%%%
% @requires
% 1. VSCode extension: LaTeX Workshop by James Yu
% 2. Command Palette (`View` > `Command Palette`) `Preferences: Open Keyboard Shortcuts (JSON)`
% 
% [
%   {
%     "key": "ctrl+b",
%     "command": "editor.action.insertSnippet",
%     "when": "editorLangId == 'latex'",
%     "args": {
%       "snippet": "\\textbf{${TM_SELECTED_TEXT}}"
%     }
%   },
%   {
%     "key": "ctrl+i",
%     "command": "editor.action.insertSnippet",
%     "when": "editorLangId == 'latex'",
%     "args": {
%       "snippet": "\\textit{${TM_SELECTED_TEXT}}"
%     }
%   }
% ]
% 
% 
% 3. Command Palette (`Ctrl` + `Shift` + `P`) `Preferences: Open User Settings (JSON)`, global settings
% 
% "latex-workshop.latex.synctex": 1
% 
% @usage
% 1. import configuration: 
%     ```
%     \input{preamble.tex}
%     ```
% 2. bold: `Ctrl` + `B`
% 3. italic: `Ctrl` + `I`
% 4. PDF Backtracking LaTeX Code: `Ctrl` + `Click`
%%%%%%%%%%%%%%%%%%%%%%%%%%%%%%%%%%%%%%%%%%%%%%%%%%%%%%%%%%%%%%%%%%%%%%%%%%%%%%

% system - comments
\usepackage{comment}                                % \begin{comment}...\end{comment}

% layout - margins
\usepackage{geometry}
\geometry{left=36pt, right=36pt, top=72pt, bottom=72pt}     % 1 in = 72 pt

% font - colors
\usepackage{xcolor}
\definecolor{redcolor}{rgb}{1.0, 0.0, 0.0}          % {\color{redcolor}...}
\definecolor{greencolor}{rgb}{0.0, 1.0, 0.0}        % {\color{greencolor}...}
\definecolor{bluecolor}{rgb}{0.0, 0.0, 1.0}         % {\color{bluecolor}...}
\definecolor{yellowcolor}{rgb}{1.0, 1.0, 0.0}       % {\color{yellowcolor}...}
\definecolor{cyancolor}{rgb}{0.0, 1.0, 1.0}         % {\color{cyancolor}...}
\definecolor{magentacolor}{rgb}{1.0, 0.0, 1.0}      % {\color{magentacolor}...}
\definecolor{blackcolor}{rgb}{0, 0, 0}              % {\color{blackcolor}...}
\definecolor{darkgreen}{RGB}{0,128,0}

% font - size
                                                    % \tiny This is tiny text. \normalsize
                                                    % \scriptsize This is scriptsize text. \normalsize
                                                    % \footnotesize This is footnotesize text. \normalsize
                                                    % \small This is small text. \normalsize
                                                    % \normalsize This is normalsize text. \normalsize
                                                    % \large This is large text. \normalsize
                                                    % \Large This is Large text. \normalsize
                                                    % \LARGE This is LARGE text. \normalsize
                                                    % \huge This is huge text. \normalsize
                                                    % \Huge This is Huge text. \normalsize

% paragraph - indentation
\setlength{\parindent}{0pt}                         % default: 18pt





% resource - hyperlinks


% resource - hyperlinks
\usepackage{hyperref}                               % \href{URL}{URL}.

% \href{
%     https://www.kaggle.com/datasets/kyanyoga/sample-sales-data
% }{Sample Sales Data}


% resource - images
\usepackage{graphicx}                               % \insertimage{size}{images/filename}{caption}
\newcommand{\insertimage}[3]{                           % \ref{fig:images/filename}
    \begin{figure}[h!]
        \centering
        \includegraphics[width=#1\textwidth]{#2}
        \caption{#3}
        \label{fig:#2}
    \end{figure}
}

% resource - videos
\usepackage{media9}                                 % \insertvideo{width}{height}{video path}{caption}{autoplay (true/false)}{loop (true/false)}
\newcommand{\insertvideo}[6]{                           % \ref{fig:video path}
    \begin{figure}[h!]
        \centering
        \includemedia[
            width=#1\textwidth,        % Video width (relative to text width)
            height=#2\textwidth,       % Video height (relative to text width)
            activate=onclick,          % Activate the video on click
            addresource=#3,            % Video file path (e.g., video.mp4)
            flashvars={
                src=#3 % Video source
                &autoPlay=#5 % Autoplay true/false
                &loop=#6     % Loop true/false
            }
        ]{}{VPlayer.swf}
        \caption{#4}                   % Caption for the video
        \label{fig:#3}                 % Label using the video path for easy referencing
    \end{figure}
}

% paragraph - code
\usepackage{listings}                               % \begin{lstlisting}\end{lstlisting}
\lstset{
  language=Python,
  commentstyle=\color{darkgreen},
  keywordstyle=\color{blue},
  stringstyle=\color{darkgreen},
  basicstyle=\ttfamily,
  breaklines=true,
  numberstyle=\tiny\color{gray},
  frame=single,
  literate=*{0}{{{\color{blue}0}}}1
            {1}{{{\color{blue}1}}}1
            {2}{{{\color{blue}2}}}1
            {3}{{{\color{blue}3}}}1
            {4}{{{\color{blue}4}}}1
            {5}{{{\color{blue}5}}}1
            {6}{{{\color{blue}6}}}1
            {7}{{{\color{blue}7}}}1
            {8}{{{\color{blue}8}}}1
            {9}{{{\color{blue}9}}}1
}



% Tabular Data (Tables)
\usepackage{array}
\usepackage{lipsum}
\usepackage{longtable}












%%%%%%%%%%%%%%%%%%%%%%%%%%%%%%%%%%%%%%%%%%%%%%%%%%%%%%%%%%
%%%%%%%%%%%%%%%%%%%%% CUSTOM DOCUMENT %%%%%%%%%%%%%%%%%%%%
%%%%%%%%%%%%5%%%%%%%%%%%%%%%%%%%%%%%%%%%%%%%%%%%%%%%%%%%%%
% title pages
\usepackage{outline} 
\usepackage{pmgraph} 
\usepackage[normalem]{ulem}
\usepackage{verbatim}
\usepackage{pgffor}
\usepackage{xstring}
\usepackage{xparse}
% Signature Fields for Agreement
\usepackage{textcomp} % for \TextField
\usepackage{ifthen}   % for conditional checks
\newcounter{nameCounter}
\newcommand{\signField}[1][Printed Name]{
    \stepcounter{nameCounter}
    \begin{tabbing}
        \hspace{0.3\textwidth} \= \hspace{0.35\textwidth} \= \hspace{0.20\textwidth} \= \hspace{0.15\textwidth} \kill
        \makebox[0.3\textwidth]{} \> \TextField[name=name\arabic{nameCounter},width=0.35\textwidth,bordercolor=0 0 0]{} \> \TextField[name=date\arabic{nameCounter},width=0.20\textwidth,bordercolor=0 0 0]{} \> \makebox[0.15\textwidth]{} \\[-8pt]
        \underline{\hspace{0.3\textwidth}} \> \underline{\hspace{0.35\textwidth}} \> \underline{\hspace{0.20\textwidth}} \> \underline{\hspace{0.15\textwidth}} \\ 
        \textit{Signature} \> #1 \> Date \> \makebox[0.15\textwidth][r]{\textbf{\textit{Initials}}}
    \end{tabbing}
}
                                                    % \signField                (default)
                                                    % \signField[custom_name]


%%%%%%%%%%%%%%%%%%%%%%%%%%%%%%%%%%%%%%%%%%%%%%%%%%%%%%%%%
% Title Page
%%%%%%%%%%%%%%%%%%%%%%%%%%%%%%%%%%%%%%%%%%%%%%%%%%%%%%%%%
\newcommand{\convertarraytostring}[2]{
  \gdef#1{}
  \foreach \x in {#2} {
    \xdef#1{#1 \x \\}
  }
}
\newcommand{\customtitle}[7]{
    \title{
        \includegraphics[width=1.75in]{images/emblem.png} \\ % fixed Logo settings
        \vspace*{0.5in}
        #1 \\ 
        \vspace*{0.5in}
        \textbf{#2}
    }
    \author{
        #3 \\ 
        \vspace*{1pt} \\
        #4 \\
        \vspace*{0.5in} \\
        \textbf{#5} \\ 
        \vspace*{1pt} \\
        #6 \\ 
        \vspace*{1pt} \\
        \textbf{#7} \\ 
    }
    \date{\today} % fixed date settings
    \maketitle
    \newpage
}

% \convertarraytostring
% {\titleauthors}                                                                       % Authors
% {Jaehoon Song (lead), Manya Jain, Devika Papal, Yashman P Singh}
% \customtitle
% {Final Team Project: Process Book}                                                    % Title
% {Music Trends and Insights: Visualization of Genres, Artists, and Listener Behaviors} % Subtitle
% {Team: Vizualytics}                                                                   % Team name (e.g., "Team: Vizualytics")
% {\titleauthors}
% {CS-4460-B: Introduction to Information Visualization}                                % Course name (e.g., "CS-4460-B: Introduction to ...")
% {Georgia Institute of Technology}                                                     % Institution (e.g., "Georgia Institute of Technology")
% {Professor: Yalong Yang}                                                              % Professor name (e.g., "Yalong Yang")












%%%%%%%%%%%%%%%%%%%%%%%%%%%%%%%%%%%%%%%%%%%%%%%%%%%%%%%%%
% First Page Handout Title
%%%%%%%%%%%%%%%%%%%%%%%%%%%%%%%%%%%%%%%%%%%%%%%%%%%%%%%%%

% General Handout Format
% \handout{<title>}{<course name>}{<date>}{<author>}{<secondary info>}{<number>}
% Creates a formatted header for handouts. Parameters:
% 1. <title>: Title of the handout (e.g., "Lecture 1")
% 2. <course name>: Name of the course (e.g., "CS 4510 Automata and Complexity")
% 3. <author>: Name of the author (e.g., "Prof. Alice")
% 4. <date>: Date of the handout (e.g., "Jan 15, 2025")
% 5. <secondary info>: Additional info (e.g., "Scribe(s): Bob and Charlie")
% 6. <number>: Handout number or identifier (used for page numbering)
\usepackage{amssymb,amsmath, tikz} % Math symbols and environments
\newcommand{\handout}[6]{
   \renewcommand{\thepage}{#6-\arabic{page}} % Sets page numbering style to <number>-<page>
   \noindent
   \begin{center}
   \framebox{
      \vbox{
         \hbox to 5.78in { #2 \hfill #3 } % Course name and date
         \vspace{4mm}
         \hbox to 5.78in { {\Large \hfill #1 \hfill} } % Title centered
         \vspace{2mm}
         \hbox to 5.78in { {\it #4 \hfill #5} } % Author and secondary info
      }
   }
   \end{center}
   \vspace*{4mm}
}

% Example Usages:
% % 1. Lecture Handout
% \handout
%    {Lecture 1}
%    {CS 4510 Automata and Complexity}{Prof. Alice}
%    {Jan 15, 2025}{Scribe(s): Bob and Charlie}
%    {1}

% % 2. Homework Handout (CS 4510 / CS 4540)

% \handout
% {Homework 1: Finite Automata}
% {\textbf{CS 4510 Automata and Complexity}}{Out: 01/08/2025}
% { }{Due: 01/22/2025}
% {1: Finite Automata}

% \handout
% {Problem Set 1} 
% {\textsc{Georgia Tech} S'25}{CS 4540: \textsc{Advanced Algorithms}}
% {Out: Jan 7}{Due: Jan 14 11.59 pm}
% {CS 4540}


% % 3. Exam Handout
% \handout
%    {Exam 1}
%    {CS 4510 Automata and Complexity}{Dr. Smith}
%    {Feb 5, 2025}{Duration: 3 hours}
%    {1}

% 4. Project Proposal
% \handout
%    {Project Proposal}
%    {CS 4510 Automata and Complexity}{Prof. Alice}
%    {Mar 1, 2025}{Team: Alpha Group}
%    {1}

% % 5. General Announcement
% \handout
%    {Midterm Announcement}
%    {CS 4510 Automata and Complexity}{Prof. Alice}
%    {Mar 10, 2025}{--}
%    {--}


% Answer environment for responses
% \answer[<color>] creates an answer environment with optional text color (default: teal).
\newenvironment{answer}[1][redcolor]
  {\newline\newline\color{#1}\noindent\textbf{Answer}: }
  {\par\hfill$\square$\newline}

% Prove environment for responses
% \prove[<color>] creates an answer environment with optional text color (default: teal).
\newenvironment{prove}[1][redcolor]
  {\newline\newline\color{#1}\noindent\textbf{Answer}: }
  {\par\hfill$\blacksquare$\newline}
  
% \begin{answer}
%     <YOUR WORK GOES HERE> s
% \end{answer}

% \begin{prove}
%     <YOUR WORK GOES HERE> s
% \end{prove}




%%%%%%%%%%%%%%%%%%%%%%%%%%%%%%%%%%%%%%%%%%%%%%%%%%%%%%%%%
% Bibliographies (with Citations)
%%%%%%%%%%%%%%%%%%%%%%%%%%%%%%%%%%%%%%%%%%%%%%%%%%%%%%%%%
\usepackage[backend=biber,style=apa]{biblatex}      % \newpage\printbibliography\end{document}
\addbibresource{references.bib}                         % \cite{reference1}

%%%%%%%%%%%%%%%%%%%%%%%%%%%%%%%%%%%%%%%%%%%%%%%%%%%%%%%%%
% DEFAULT
%%%%%%%%%%%%%%%%%%%%%%%%%%%%%%%%%%%%%%%%%%%%%%%%%%%%%%%%%
                                                    % \textbf{boldfaced}
                                                    % \textit{intalicized}